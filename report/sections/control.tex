\section{Control Design}

\begin{figure}[H]
\centering
\includegraphics[width=\textwidth]{diagram.png}
\caption{Controller block diagram}\label{fig:diagram}
\end{figure}

\noindent Figure~\ref{fig:diagram} shows the controller block diagram in the dashed box.

\subsection{Integral Action Model}
The state-space model was augmented by two integral states, $x_i = {\begin{bmatrix}\smallint{E} & \smallint{\Theta}\end{bmatrix}}^T$. Since integral action is required for zero-offset tracking, these are required to meet the specification. It was found that integral action for the pitch axis was not necessary, as it settles to equilibrium with a zero input. The integral states are with reference to a setpoint, so the references for the corresponding position states are subtracted before integration through the discrete-time integrator blocks. This was done explicitly in the Simulink model, as adding them to $C$ was required for system observability. Rather than splitting into separate integral action and state feedback gains, a single state-space system and feedback gain was chosen for convenience with trying other control architectures and script simulations. 

\subsection{Kalman Filter Tuning}
To characterise the noise from the measurements in $C$, a sample of data was taken of the helicopter at rest, unforced. After removing bias, the MATLAB \texttt{cov} function was used to automatically calculate $R$ in the Kalman Filter. To characterise the process noise, a test log was used to compare against a linear simulation with the same input series---taking the covariance of the differences. This was more difficult as the test logs need to be smooth and accurate, a median filter and a moving mean filter (offset for the induced lag) were used in conjunction. Though these tests gave a starting point, $Q$ was further adjusted by hand with tests. 

\subsection{Reference Generation}
Two reference generators were made: a finite impulse response (FIR) filter method and a nonlinear optimisation method. These can be selected through Simulink Variants. 

\subsubsection{FIR Method}
A second-order FIR filter was developed for the $\Theta$ and $\dot{\Theta}$ reference as the feasible option for real-time computation on an embedded system. Using the Lecture 10 notes as a reference, the travel axis reference is user-parameterised by the transition time $T_{\mathrm{trans}}$ and maximum desired pitch angle $\Psi_{\max}$. The method in the notes uses $A_{\max}$, which can be made a function of $\Psi$ to be more intuitive. The maximum travel axis angular acceleration that can be achieved for a given pitch angle is limited by the maximum force of the fans, $2 F_{\max}$, at the distance $l_F$ (derived earlier), and the inertia about the axis, $J_{\Theta}$.
\[
\alpha_{\max} (\Psi) = \frac{2 F_{\max} \, \sin(\Psi) \, l_F}{J_{\Theta}}
\]
\noindent Given a velocity calculated from transition displacement (180$^{\circ}$) and time, $v_{\mathrm{trans}} = s_{\mathrm{trans}} / T_{\mathrm{trans}}$, a smoothing kernel of time-width $T_{\mathrm{kern}} = v_{\mathrm{trans}} / \alpha_{\max}$ is twice convolved with the rectangular velocity profile. The position trajectory is the integral of this velocity profile. 

The FIR reference makes the assumption that the maximum pitch angle and maximum fan force can be achieved instantly; therefore, it is not precisely feasible, but it is a smooth reference and can be computed very quickly.

\subsubsection{Nonlinear Optimisation Method}
Using the nonlinear dynamic model derived from the Euler-Lagrange modelling and a first-order approximation of the fan response, full-state optimal transition trajectories were computed with a nonlinear program (NLP) using direct collocation in \href{http://www.ee.ic.ac.uk/ICLOCS/default.htm}{ICLOCS2}. The continuous-time optimisation problem was formulated as
\[
\setlength{\jot}{8pt}% tweak
\begin{aligned}
  \min_{U_a, \, U_b} \qquad  & \int_{0}^{t_f} {U_a(t)}^2 + {U_b(t)}^2 \, dt \\
  \mbox{s.t.} \qquad & \dot{x}(t) = f(t, \, x(t), \, u(t)) \\
  \,                & -10^{\circ} \le E(t) \le 10^{\circ}, \quad -20^{\circ} \le \Theta(t) \le 200^{\circ} \\
%   \,                & -20^{\circ} \le \Theta(t) \le 200^{\circ} \\
  \,                & E(0) = E(t_f) = \Theta(0) = 0^{\circ}, \quad \Theta(t_f) = 180^{\circ} \\
%   \,                & E(0) = 0^{\circ}, \quad \Theta(0) = 0^{\circ} \\
%   \,                & E(t_f) = 0^{\circ}, \quad \Theta(t_f) = 180^{\circ} \\
\end{aligned}
\label{eq:opt}
\]
The final time $t_f$ is left free as it was found that the solution transitions within a good margin of the specification. The above formulation is for the outward trajectory; for the return, the boundary constraints on $\Theta$ are reversed. 

The result is a robust and dynamically feasible reference with no input saturation, however, implementing nonlinear optimisation in a real-time embedded system is dubious, as the NLP is slow and is not guaranteed to converge to a solution. Since the embedded implementation is not given as a priority, this method will be used for the results to maximise the performance.

\subsection{Feedback and Feedforward Gains}
Using the model with augmented integral states, an LQR feedback gain, $K_{\mathrm{LQR}}$, was designed. The model simulation was tuned such that the trajectory for the same controller matched the test reasonably. Cost tuning could be performed more quickly with the simulated model; once a good level of control was achieved, the costs were fine-tuned to the physical model. Using the optimised trajectory, it was found that $R$ could be set very low as the reference is already optimised for minimum effort, allowing it to track closely throughout the transition. Minimal cost in $Q$ was set on the pitch axis, using only the pitch rate cost and reference to get the system moving at the beginning of the transition. High cost is set on the travel rate to achieve minimal overshoot and oscillation. As minimal oscillation and overshoot was experienced on the elevation axis, it was not costed highly overall---the integral action does most of the work to maintain tracking. The feedforward gain $K_r$ was calculated using the standard formula.


